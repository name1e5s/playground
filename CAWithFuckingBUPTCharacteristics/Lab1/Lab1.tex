% Koko
\documentclass[blue,normal,cn]{elegantnote}
\usepackage{array}
\usepackage{courier}
\usepackage{xcolor}
\usepackage{zhnumber}
\usepackage{ulem}
\usepackage{float}

\definecolor{light-gray}{gray}{0.95}
\newcommand{\code}[1]{\colorbox{light-gray}{\texttt{#1}}}
\newfontfamily\courier{Courier New}
\lstset{linewidth=1.1\textwidth,
	numbers=left,
	basicstyle=\small\courier,
	numberstyle=\tiny\courier,
	keywordstyle=\color{blue}\courier,
	commentstyle=\it\color[cmyk]{1,0,1,0}\courier, 
	stringstyle=\it\color[RGB]{128,0,0}\courier,
	frame=single,
	backgroundcolor=\color[RGB]{245,245,244},
	breaklines,
	extendedchars=false, 
	xleftmargin=2em,xrightmargin=2em, aboveskip=1em,
	tabsize=4, 
	showspaces=false
	basicstyle=\small\courier
}
\title{实验 1: MIPS 指令系统和 MIPS 体系结构}
\version{$\aleph$}
\date{\zhtoday}

\begin{document}
\author{
    \begin{tabular}[t]{c}
        于海鑫 \\
        2017211240
    \end{tabular}
}
\maketitle

\section{实验目的}
\begin{enumerate}
    \item 了解和熟悉指令级模拟器
    \item 熟练掌握 MIPSsim 模拟器的操作和使用方法
    \item 熟悉 MIPS 指令系统及其特点,加深对 MIPS 指令操作语义的理解
    \item 熟悉 MIPS 体系结构
\end{enumerate}

\section{实验平台}

实验平台采用指令级和流水线操作级模拟器 \code{MIPSsim}。

\section{实验内容和步骤}

首先要阅读 MIPSsim 模拟器的使用方法(见附录),然后了解 MIPSsim 的指令系统和汇编语言。

\begin{enumerate}[wide=0pt, listparindent=2em, parsep=0pt]
    \item 启动 MIPSsim(用鼠标双击 MIPSsim.exe)。
    \item 选择 “配置” $\rightarrow$ “流水方式” 选项,使模拟器工作在非流水方式下。
    \item 参照 MIPSsim 使用说明,熟悉 MIPSsim 模拟器的操作和使用方法。

          可以先载人一个样例程序(在本模拟器所在的文件夹下的“样例程序”文件夹中),然后
          分别以单步执行一条指令、执行多条指令、连续执行、设置断点等的方式运行程序,观
          察程序执行情况,观察 CPU 中寄存器和存储器的内容的变化。
    \item 选择“文件” $\rightarrow$ “载入程序” 选项,加载样例程序 \code{alltest.asm} ,然后查看 “代码” 窗口,至看程序所在的位置(起始地址为 \code{0x00000000})

          \begin{figure}[H]
              \centering
              \includegraphics[width=1\textwidth]{fig/load_alltest.png}
              \caption{载入程序}
              \label{fig:load_alltest}
          \end{figure}

    \item 查看 “寄存器” 窗口 PC 寄存器的值:\textbf{[PC]} = \uline{0x00000000}。
    \item 执行 \code{load} 和 \code{store} 指令,步骤如下:

          \begin{itemize}[leftmargin=3em]
              \item 单步执行 1 条指令(F7)。
              \item 下一条指令地址为 \uline{0x00000004},是一条 \uline{有} 符号载入 \uline{字节} 指令。
              \item 单步执行 1 条指令(F7)。
              \item 查看 R1 的值,\textbf{[R1]} = \uline{0xFFFFFFFFFFFFFF80}。
              \item 下一条指令地址为 \uline{0x00000008},是一条 \uline{无} 符号载入 \uline{字} 指令。
              \item 单步执行 1 条指令(F7)。
              \item 查看 R1 的值,\textbf{[R1]} = \uline{0x0000000000000080}。
              \item 下一条指令地址为 \uline{0x0000000C},是一条 \uline{无} 符号载入 \uline{字节} 指令。
              \item 单步执行 1 条指令(F7)。
              \item 查看 R1 的值,\textbf{[R1]} = \uline{0x0000000000000080}。
              \item 单步执行 1 条指令(F7)。
              \item 下一条指令地址为 \uline{0x00000014},是一条保存 \uline{字} 指令。
              \item 单步执行 1 条指令(F7)。
              \item 查看内存 \code{BUFFER} 处字的值,值为 \uline{0x80}。(内存 $\rightarrow$ 符号表)
          \end{itemize}

    \item 执行逻辑运算类指令。步骤如下:
          \begin{itemize}[leftmargin=3em]
              \item 双击 “寄存器” 窗口中的 R1,将其值修改为 2
                    。
              \item 双击 “寄存器” 窗口中的 R2,将其值修改为 3
              \item 单步执行 1 条指令。
              \item 下一条指令地址为 \uline{0x00000020},是一条加法指令。
              \item 单步执行 1 条指令。
              \item 查看 R3 的值,\textbf{[R3]} = \uline{0x0000000000000005}。
              \item 下一条指令地址为 \uline{0x00000024},是一条乘法指令。
              \item 单步执行 1 条指令。
              \item 查看 L0、HI 的值,\textbf{[LO]}= \uline{0x0000000000000006},\textbf{[HI]}= \uline{0x0000000000000000}。
          \end{itemize}

    \item 执行逻辑运算类指令。步骤如下:

          \begin{itemize}[leftmargin=3em]
              \item 双击 “寄存器” 窗口中的 R1,将其值修改为 0xFFFF0000。
              \item 双击 “寄存器” 窗口中的 R2,将其值修改为 0xFF00FF00。
              \item 单步执行 1 条指令。
              \item 下一条指令地址为 \uline{0x00000030},是一条逻辑与运算指令,第二个操作数寻址方式是 \uline{寄存器直接寻址}。
              \item 单步执行 1 条指令。
              \item 查看 R3 的值,\textbf{[R3]} = \uline{0x00000000FF000000}。
              \item 下一条指令地址为 \uline{0x00000034},是一条逻辑与运算指令,第二个操作数寻址方式是 \uline{立即数寻址}。
              \item 单步执行 1 条指令。
              \item 查看 R3 的值,\textbf{[R3]} = \uline{0x0000000000000000}。
          \end{itemize}
    \item 执行控制转移类指令。步骤如下:
          \begin{itemize}[leftmargin=3em]
              \item 双击 “寄存器” 窗口中的 R1,将其值修改为 2
                    。
              \item 双击 “寄存器” 窗口中的 R2,将其值修改为 2
              \item 单步执行 1 条指令。
              \item 下一条指令地址为 \uline{0x00000040},是一条 BEQ 指令,其测试条件是 \uline{\$r1 == \$r2},目标地址为 \uline{0x0000004C}。
              \item 单步执行 1 条指令。
              \item 查看 PC 的值,\textbf{[PC]}= \uline{0x0000004C},表明分支 \uline{成功}。
              \item 下一条指令是一条 BGEZ 指令,其测试条件是 \uline{\$r1 $\geq 0$},目标地址为 \uline{0x00000058}。
              \item 单步执行 1 条指令。
              \item 查看 PC 的值,\textbf{[PC]}= \uline{0x00000058},表明分支 \uline{成功}。
              \item 下一条指令是一条 BGEZAL 指令,其测试条件是 \uline{\$r1 $\geq 0$},目标地址为 \uline{0x00000064}。
              \item 单步执行 1 条指令。
              \item 查看 PC 的值,\textbf{[PC]}= \uline{0x00000064},表明分支 \uline{成功};查看 R31 的值,\textbf{[R31]} = \uline{0x0000005C}。
              \item 单步执行 1 条指令。
              \item 查看 R1 的值,\textbf{[R1]} = \uline{0x0000000000000074}。
              \item 下一条指令地址为 \uline{0x00000068},是一条 JALR 指令,保存目标地址的寄存器为 R\uline{1},保存返回地址的目标寄存器为 R\uline{3}。
              \item 单步执行 1 条指令。
              \item 查看 PC 和 R3 的值,\textbf{[PC]}= \uline{0x00000074},[R3]=\uline{0x000000000000006C}。
          \end{itemize}
          \begin{figure}[H]
              \centering
              \includegraphics[width=1\textwidth]{fig/fin_alltest.png}
              \caption{程序运行完成}
              \label{fig:fin_alltest}
          \end{figure}
\end{enumerate}

\section{实验中的问题与心得}

本次实验主要目的是在于熟悉模拟器的使用以及 MIPS 体系结构的特点,实验中没有遇到任何问题。

至于心得,个人认为本次实验中的填空过于繁琐了,PC 的位置无需重复确认,其行为在 MIPS ISA 的手册上有着明确定义,只通过展示正常情况下的 PC 变化以及发生跳转时 PC 的变化即可使得同学们较好的了解和把握 MIPS 体系结构的特点。同理,对于指令作用的重复提问也很繁琐且没有必要。

\end{document}